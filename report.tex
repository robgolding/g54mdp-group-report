\documentclass[a4paper,11pt]{article}

\usepackage{url}
\usepackage{graphicx}
\usepackage{enumerate}
\usepackage{amsmath}
\usepackage{float}
\usepackage{longtable}
%\usepackage{fullpage}
\usepackage{pstricks}
\usepackage{tikz}
\usepackage[absolute]{textpos}
\usepackage{import}
\usepackage{subfigure}
\usepackage{setspace}

\title{G54MDP Individual Report}
\author{Robert J. Golding (rjg08u)} \date{\today}

% Dutch style paragraph formatting
\setlength{\parskip}{1.3ex plus 0.2ex minus 0.2ex}
\setlength{\parindent}{0pt}

%\doublespacing
\onehalfspacing

\begin{document}
    \maketitle

    \section{Introduction}

    Our group has created a news feed/RSS reader for iOS devices. The feed
    reader allows users to easily add feeds using a search function (so no
    typing in long feed URLs), and read news items with an embedded Safari
    browser. Also, all feeds can be viewed at once, interleaving items with the
    most recent at the top.
    
    Originally, the application was to be very similar to the Pulse newsreader
    application (which is described in detail further in this report) but the
    design diverged as the project continued.

    Though this project started life as a networked multi-player game, it was
    decided relatively early on in the process that a less complex, more
    standard application was preferable. The group felt that it would be easier
    to showcase our ability to create a good mobile application with a simpler
    project brief---hence the feed reader proposal.

    The work was split over the four group members as follows:

    \paragraph{Rob Golding}

    \begin{itemize}
        \item Web service feasibility testing
        \item Evaluation of existing services
    \end{itemize}

    \paragraph{Michal Konturek}

    \begin{itemize}
        \item Final application development
        \item Implementation of proposed interface design
    \end{itemize}

    \paragraph{Rob Miles}

    \begin{itemize}
        \item Research into feed parsing technologies
        \item Application architecture proposal
    \end{itemize}

    \paragraph{Marcus Whybrow}

    \begin{itemize}
        \item Interface design
        \item Initial iOS prototype development
    \end{itemize}

    The initial research into both existing applications and suitable
    technologies was conducted by both myself and Rob Miles. The Pulse
    newsreader application was considered to be the most similar in design to
    the result the group wished to achieve.

    \section{Pulse}

    \emph{Pulse} uses a custom web service to provide extra functionality to
    users, such as a choice of feeds to add, automatic synchronisation, and
    rendered text display. Also, ``featured'' feeds can be displayed which,
    presumably, publishers would be willing to pay for. The custom web service
    model allows more complex logic to be included in the application without
    bloating the mobile client itself.

    The main interface of the Pulse application comprises a number of
    horizontal feeds displayed in a vertical list, each with its own scrolling
    capability. Over-scrolling downwards synchronises all feeds using the
    familiar iPhone ``arrow'' animation which users have come to expect. When
    selecting an item, the ``summary'' of the item is displayed in a rendered
    text display screen. For the full detail, an embedded browser is used to
    show the web page linked to by the item.

    \section{Technologies}

    To implement our application, the group had to make a decision between
    using a similar architecture to Pulse---using a custom web service
    ``middle-man''---or adding the logic to the app itself, negating the need
    for a custom service.

    In the end, a compromise was reached between the two methods. Through
    research conducted by Rob Miles and myself, it was discovered that Google
    offers a \emph{feeds API}, which is capable of parsing an RSS or ATOM feed,
    and emitting a JSON format response which can be easily parsed by a mobile
    application. Also, the feeds API allows searching by keywords, and returns
    the results of that search, again in JSON format. This meant that the
    functionality that could be gained from implementing a custom web service
    was already available in the form of a free, publicly available API.

    \section{Architecture}

    The mobile application, it was decided, would make heavy use of the Google
    feeds API. Users would be able to add feeds from lists of pre-determined
    categories, populated by querying the Feeds API for the category name. For
    example, ``middle east'', ``USA'' or ``politics'' might each represent
    a category of feeds, with the user then able to choose from a list of
    results to add to their personal feeds list.
    
    The feeds would then be checked periodically using the same API, by parsing
    the JSON data returned. JSON is smaller, faster and uses less memory to
    parse than XML, and so is more suitable for a mobile platform.  Also, the
    feeds API guarantees that the data will be valid---so data validation is
    less of a concern.

    \section{The Group}

    Regular meetings were held every week for group decision making during term
    time, when we produced a basic specification for our application. After
    this time, we had several coding sessions to allow the app to come into
    fruition.
    
    The group were mostly in the same place making this relatively easy, though
    sometimes we were geographically separated. In this case, we made use of
    Skype whilst meetings were taking place to ensure that all group members
    were involved in the process if making major decisions. At other times, we
    used instant messaging for more informal communications.

    Finally, the project was managed using the Basecamp project management
    suite, which we used to keep track of deadlines and to keep up-to-date
    between meetings.

    \section{Personal Contribution}

    In contribution to this project, I have conducted research into the field
    including the architecture of related applications such as the
    aforementioned Pulse newsreader. Also, I spent some time creating
    a proof-of-concept web service which provided basic feed parsing
    capability. The web service was originally intended to store user
    preferences such as the list of subscribed feeds, though it was decided
    that this functionality should be moved into the application itself once
    the Google feeds API was discovered. I also discovered a number of iOS
    libraries which could be used to replicate this functionality on the mobile
    device itself---though in the end this was not required.

    Based upon this research, I contributed to the design of the application's
    architecture by applying the knowledge gained in this area. Also,
    I investigated and documented the Google feeds API features which would be
    required for the project, so that the application could be developed more
    quickly and easily. Though I am, by no means, an experienced iOS developer,
    I felt I was able to contribute to the project in a significant way through
    this type of research and proof-of-concept development.

    \section{Conclusion}

    Though at times apart, the group performed very well and has produced
    a feasible, usable application. The tasks were distributed evenly between
    group members, and then progress organised efficiently through Basecamp.

    I feel that the group worked very well together, and was able to create
    a successful application in the time allotted. Specifically, the group
    members could draw on the mutual respect and trust which exists between
    each other. This is something which I consider essential for a successful
    group task, but can take a significantly long time to achieve and is
    therefore so often absent.

    For this reason, I feel privileged to have worked with such talented,
    committed individuals, and would relish the opportunity to do so again.

    In my opinion, the final application represents a smooth and usable
    experience for the end user, and minimises the effort of searching for and
    adding new feeds. Also, synchronisation is fast due to the underlying JSON
    format which offers small, fast data encoding.

\end{document}
